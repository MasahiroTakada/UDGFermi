\documentclass[a4paper,11pt]{article}
\pdfoutput=1 % if your are submitting a pdflatex (i.e. if you have
             % images in pdf, png or jpg format)

\usepackage{jcappub} % for details on the use of the package, please
                     % see the JCAP-author-manual

\usepackage[T1]{fontenc} % if needed

\usepackage{url}
\usepackage{graphics}

\title{\boldmath Constraining dark matter
annihilation with HSC ultra diffuse galaxies}


%% %simple case: 2 authors, same institution
%% \author{A. Uthor}
%% \author{and A. Nother Author}
%% \affiliation{Institution,\\Address, Country}

% more complex case: 4 authors, 3 institutions, 2 footnotes
\author[a,b,1]{Daichi Hashimoto,\note{Corresponding author.}}
\author[c]{et al.}

% The "\note" macro will give a warning: "Ignoring empty anchor..."
% you can safely ignore it.

\affiliation[a]{Nagoya U.}
\affiliation[c]{A School for Advanced Studies,\\some-location, Country}

% e-mail addresses: one for each author, in the same order as the authors
\emailAdd{first@one.univ}
\emailAdd{fourth@one.univ}




\abstract{Abstract...}



\begin{document}
\maketitle
\flushbottom

\section{Introduction}
\label{sec:intro}

\section{Model}
\label{sec:model}

\begin{itemize}
    \item Describe how to compute the luminosity of DM annihilation 
    signal for ultra diffuse galaxies
    \item Describe ingredients of NFW model ($grizy$ SED fitting, mass-to-light ratio, 
    NFW profile, concentration, boost factor due to substructures)
\end{itemize}

\section{Sample (TBD)} 

\begin{itemize}
    \item Describe the Subaru Hyper Suprime-Cam survey
    \item Describe the sample of UDGs (refer Johnny's paper)
    \item Fermi data (Section 3 in \url{http://iopscience.iop.org/article/10.1088/0067-0049/199/2/31/pdf})
    \item Show the best-fit SED for each of 8 UDGs
\end{itemize}

\section{Results}

\begin{itemize}
    \item 95\% CL upper limit on the DM cross section from 
    each of 8 UDGs, and the stacked result
\end{itemize}

\section{Discussion}
\begin{itemize}
    \item Impact of model variations on the results
    \item Discuss the results when using $\sim$500 UDGs
    \item Discuss prospects for the full HSC survey and LSST
    \item Discuss pros and cons of this method compared with dSphs constraints (stress advantages of point source-like targets)
    \item Discuss implications for Magic/HESS and CTA (HAWC: TeV DM) 
\end{itemize}

\section{Conclusion}

\acknowledgments
This work is in part supported by MEXT Grant-in-Aid for Scientific Research on Innovative Areas (No.~15H05887, 15H05892, 15H05893).

% The bibliography will probably be heavily edited during typesetting.
% We'll parse it and, using the arxiv number or the journal data, will
% query inspire, trying to verify the data (this will probalby spot
% eventual typos) and retrive the document DOI and eventual errata.
% We however suggest to always provide author, title and journal data:
% in short all the informations that clearly identify a document.

\bibliography{refs}

\end{document}
