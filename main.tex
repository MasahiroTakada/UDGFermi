\documentclass[a4paper,11pt]{article}
\pdfoutput=1 % if your are submitting a pdflatex (i.e. if you have
             % images in pdf, png or jpg format)

\usepackage{jcappub} % for details on the use of the package, please
                     % see the JCAP-author-manual

\usepackage[T1]{fontenc} % if needed

\usepackage{url}
\usepackage{graphics}

\title{\boldmath Constraining dark matter
annihilation with HSC ultra diffuse galaxies}


%% %simple case: 2 authors, same institution
%% \author{A. Uthor}
%% \author{and A. Nother Author}
%% \affiliation{Institution,\\Address, Country}

% more complex case: 4 authors, 3 institutions, 2 footnotes
\author[a,b,1]{Daiki Hashimoto,\note{Corresponding author.}}
\author[c]{et al.}

% The "\note" macro will give a warning: "Ignoring empty anchor..."
% you can safely ignore it.

\affiliation[a]{Nagoya U.}
\affiliation[c]{A School for Advanced Studies,\\some-location, Country}

% e-mail addresses: one for each author, in the same order as the authors
\emailAdd{first@one.univ}
\emailAdd{fourth@one.univ}




\abstract{Abstract...}



\begin{document}
\maketitle
\flushbottom

\section{Introduction}
\label{sec:intro}

\section{Model}
\label{sec:model}

\begin{itemize}
    \item Describe how to compute the luminosity of DM annihilation 
    signal for ultra diffuse galaxies
    \item Describe ingredients of NFW model ($grizy$ SED fitting, mass-to-light ratio, 
    NFW profile, concentration, boost factor due to substructures)
\end{itemize}

\section{Sample (TBD)} 

\begin{itemize}
    \item Describe the Subaru Hyper Suprime-Cam survey
    \item Describe the sample of UDGs (refer Johnny's paper)
    \item Fermi data (Section 3 in \url{http://iopscience.iop.org/article/10.1088/0067-0049/199/2/31/pdf})
    \item Show the best-fit SED for each of 8 UDGs
\end{itemize}

\section{Results}

\begin{itemize}
    \item 95\% CL upper limit on the DM cross section from 
    each of 8 UDGs, and the stacked result
\end{itemize}

\section{Discussion}
\begin{itemize}
    \item Impact of model variations on the results
    \item Discuss the results when using $\sim$500 UDGs
    \item Discuss prospects for the full HSC survey and LSST
    \item Discuss pros and cons of this method compared with dSphs constraints (stress advantages of point source-like targets)
    \item Discuss implications for Magic/HESS and CTA (HAWC: TeV DM) 
\end{itemize}

\section{Conclusion}

\acknowledgments
This work is in part supported by MEXT Grant-in-Aid for Scientific Research on Innovative Areas (No.~15H05887, 15H05892, 15H05893).

% The bibliography will probably be heavily edited during typesetting.
% We'll parse it and, using the arxiv number or the journal data, will
% query inspire, trying to verify the data (this will probably spot
% eventual typos) and retrive the document DOI and eventual errata.
% We however suggest to always provide author, title and journal data:
% in short all the information that clearly identify a document.

The Hyper Suprime-Cam (HSC) collaboration includes the astronomical
communities of Japan and Taiwan, and Princeton University.
The HSC instrumentation and software were developed by the National
Astronomical Observatory of Japan (NAOJ), the Kavli Institute for the
Physics and Mathematics of the Universe (Kavli IPMU), the University
of Tokyo, the High Energy Accelerator Research Organization (KEK), the
Academia Sinica Institute for Astronomy and Astrophysics in Taiwan
(ASIAA), and Princeton University.  Funding was contributed by the FIRST 
program from Japanese Cabinet Office, the Ministry of Education, Culture, 
Sports, Science and Technology (MEXT), the Japan Society for the 
Promotion of Science (JSPS),  Japan Science and Technology Agency 
(JST),  the Toray Science  Foundation, NAOJ, Kavli IPMU, KEK, ASIAA,  
and Princeton University.

The Pan-STARRS1 Surveys (PS1) have been made possible through
contributions of the Institute for Astronomy, the University of
Hawaii, the Pan-STARRS Project Office, the Max-Planck Society and its
participating institutes, the Max Planck Institute for Astronomy,
Heidelberg and the Max Planck Institute for Extraterrestrial Physics,
Garching, The Johns Hopkins University, Durham University, the
University of Edinburgh, Queen's University Belfast, the
Harvard-Smithsonian Center for Astrophysics, the Las Cumbres
Observatory Global Telescope Network Incorporated, the National
Central University of Taiwan, the Space Telescope Science Institute,
the National Aeronautics and Space Administration under Grant
No. NNX08AR22G issued through the Planetary Science Division of the
NASA Science Mission Directorate, the National Science Foundation
under Grant No. AST-1238877, the University of Maryland, and Eotvos
Lorand University (ELTE).

This paper makes use of software developed for the Large Synoptic
Survey Telescope. We thank the LSST Project for making their code
available as free software at http://dm.lsst.org.

\bibliography{bibdata}

\end{document}